% Options for packages loaded elsewhere
\PassOptionsToPackage{unicode}{hyperref}
\PassOptionsToPackage{hyphens}{url}
%
\documentclass[
]{article}
\usepackage{amsmath,amssymb}
\usepackage{iftex}
\ifPDFTeX
  \usepackage[T1]{fontenc}
  \usepackage[utf8]{inputenc}
  \usepackage{textcomp} % provide euro and other symbols
\else % if luatex or xetex
  \usepackage{unicode-math} % this also loads fontspec
  \defaultfontfeatures{Scale=MatchLowercase}
  \defaultfontfeatures[\rmfamily]{Ligatures=TeX,Scale=1}
\fi
\usepackage{lmodern}
\ifPDFTeX\else
  % xetex/luatex font selection
\fi
% Use upquote if available, for straight quotes in verbatim environments
\IfFileExists{upquote.sty}{\usepackage{upquote}}{}
\IfFileExists{microtype.sty}{% use microtype if available
  \usepackage[]{microtype}
  \UseMicrotypeSet[protrusion]{basicmath} % disable protrusion for tt fonts
}{}
\makeatletter
\@ifundefined{KOMAClassName}{% if non-KOMA class
  \IfFileExists{parskip.sty}{%
    \usepackage{parskip}
  }{% else
    \setlength{\parindent}{0pt}
    \setlength{\parskip}{6pt plus 2pt minus 1pt}}
}{% if KOMA class
  \KOMAoptions{parskip=half}}
\makeatother
\usepackage{xcolor}
\usepackage[margin=1in]{geometry}
\usepackage{color}
\usepackage{fancyvrb}
\newcommand{\VerbBar}{|}
\newcommand{\VERB}{\Verb[commandchars=\\\{\}]}
\DefineVerbatimEnvironment{Highlighting}{Verbatim}{commandchars=\\\{\}}
% Add ',fontsize=\small' for more characters per line
\usepackage{framed}
\definecolor{shadecolor}{RGB}{248,248,248}
\newenvironment{Shaded}{\begin{snugshade}}{\end{snugshade}}
\newcommand{\AlertTok}[1]{\textcolor[rgb]{0.94,0.16,0.16}{#1}}
\newcommand{\AnnotationTok}[1]{\textcolor[rgb]{0.56,0.35,0.01}{\textbf{\textit{#1}}}}
\newcommand{\AttributeTok}[1]{\textcolor[rgb]{0.13,0.29,0.53}{#1}}
\newcommand{\BaseNTok}[1]{\textcolor[rgb]{0.00,0.00,0.81}{#1}}
\newcommand{\BuiltInTok}[1]{#1}
\newcommand{\CharTok}[1]{\textcolor[rgb]{0.31,0.60,0.02}{#1}}
\newcommand{\CommentTok}[1]{\textcolor[rgb]{0.56,0.35,0.01}{\textit{#1}}}
\newcommand{\CommentVarTok}[1]{\textcolor[rgb]{0.56,0.35,0.01}{\textbf{\textit{#1}}}}
\newcommand{\ConstantTok}[1]{\textcolor[rgb]{0.56,0.35,0.01}{#1}}
\newcommand{\ControlFlowTok}[1]{\textcolor[rgb]{0.13,0.29,0.53}{\textbf{#1}}}
\newcommand{\DataTypeTok}[1]{\textcolor[rgb]{0.13,0.29,0.53}{#1}}
\newcommand{\DecValTok}[1]{\textcolor[rgb]{0.00,0.00,0.81}{#1}}
\newcommand{\DocumentationTok}[1]{\textcolor[rgb]{0.56,0.35,0.01}{\textbf{\textit{#1}}}}
\newcommand{\ErrorTok}[1]{\textcolor[rgb]{0.64,0.00,0.00}{\textbf{#1}}}
\newcommand{\ExtensionTok}[1]{#1}
\newcommand{\FloatTok}[1]{\textcolor[rgb]{0.00,0.00,0.81}{#1}}
\newcommand{\FunctionTok}[1]{\textcolor[rgb]{0.13,0.29,0.53}{\textbf{#1}}}
\newcommand{\ImportTok}[1]{#1}
\newcommand{\InformationTok}[1]{\textcolor[rgb]{0.56,0.35,0.01}{\textbf{\textit{#1}}}}
\newcommand{\KeywordTok}[1]{\textcolor[rgb]{0.13,0.29,0.53}{\textbf{#1}}}
\newcommand{\NormalTok}[1]{#1}
\newcommand{\OperatorTok}[1]{\textcolor[rgb]{0.81,0.36,0.00}{\textbf{#1}}}
\newcommand{\OtherTok}[1]{\textcolor[rgb]{0.56,0.35,0.01}{#1}}
\newcommand{\PreprocessorTok}[1]{\textcolor[rgb]{0.56,0.35,0.01}{\textit{#1}}}
\newcommand{\RegionMarkerTok}[1]{#1}
\newcommand{\SpecialCharTok}[1]{\textcolor[rgb]{0.81,0.36,0.00}{\textbf{#1}}}
\newcommand{\SpecialStringTok}[1]{\textcolor[rgb]{0.31,0.60,0.02}{#1}}
\newcommand{\StringTok}[1]{\textcolor[rgb]{0.31,0.60,0.02}{#1}}
\newcommand{\VariableTok}[1]{\textcolor[rgb]{0.00,0.00,0.00}{#1}}
\newcommand{\VerbatimStringTok}[1]{\textcolor[rgb]{0.31,0.60,0.02}{#1}}
\newcommand{\WarningTok}[1]{\textcolor[rgb]{0.56,0.35,0.01}{\textbf{\textit{#1}}}}
\usepackage{graphicx}
\makeatletter
\def\maxwidth{\ifdim\Gin@nat@width>\linewidth\linewidth\else\Gin@nat@width\fi}
\def\maxheight{\ifdim\Gin@nat@height>\textheight\textheight\else\Gin@nat@height\fi}
\makeatother
% Scale images if necessary, so that they will not overflow the page
% margins by default, and it is still possible to overwrite the defaults
% using explicit options in \includegraphics[width, height, ...]{}
\setkeys{Gin}{width=\maxwidth,height=\maxheight,keepaspectratio}
% Set default figure placement to htbp
\makeatletter
\def\fps@figure{htbp}
\makeatother
\setlength{\emergencystretch}{3em} % prevent overfull lines
\providecommand{\tightlist}{%
  \setlength{\itemsep}{0pt}\setlength{\parskip}{0pt}}
\setcounter{secnumdepth}{-\maxdimen} % remove section numbering
\ifLuaTeX
  \usepackage{selnolig}  % disable illegal ligatures
\fi
\usepackage{bookmark}
\IfFileExists{xurl.sty}{\usepackage{xurl}}{} % add URL line breaks if available
\urlstyle{same}
\hypersetup{
  pdftitle={DATA607Project2},
  pdfauthor={Erick Hadi},
  hidelinks,
  pdfcreator={LaTeX via pandoc}}

\title{DATA607Project2}
\author{Erick Hadi}
\date{2024-10-13}

\begin{document}
\maketitle

\subsection{Task}\label{task}

\begin{enumerate}
\def\labelenumi{(\arabic{enumi})}
\tightlist
\item
  Choose any three of the ``wide'' datasets identified in the Week 6
  Discussion items. (You may use your own dataset; please don't use my
  Sample Post dataset, since that was used in your Week 6 assignment!)
  For each of the three chosen datasets:  Create a .CSV file (or
  optionally, a MySQL database!) that includes all of the information
  included in the dataset. You're encouraged to use a ``wide'' structure
  similar to how the information appears in the discussion item, so that
  you can practice tidying and transformations as described below. 
  Read the information from your .CSV file into R, and use tidyr and
  dplyr as needed to tidy and transform your data. {[}Most of your grade
  will be based on this step!{]}  Perform the analysis requested in the
  discussion item.  Your code should be in an R Markdown file, posted
  to rpubs.com, and should include narrative descriptions of your data
  cleanup work, analysis, and conclusions.
\item
  Please include in your homework submission, for each of the three
  chosen datasets:  The URL to the .Rmd file in your GitHub repository,
  and  The URL for your rpubs.com web page.
\end{enumerate}

\begin{Shaded}
\begin{Highlighting}[]
\FunctionTok{library}\NormalTok{(stringr)}
\FunctionTok{library}\NormalTok{(tidyr)}
\FunctionTok{library}\NormalTok{(dplyr)}
\end{Highlighting}
\end{Shaded}

\begin{verbatim}
## 
## Attaching package: 'dplyr'
\end{verbatim}

\begin{verbatim}
## The following objects are masked from 'package:stats':
## 
##     filter, lag
\end{verbatim}

\begin{verbatim}
## The following objects are masked from 'package:base':
## 
##     intersect, setdiff, setequal, union
\end{verbatim}

\begin{Shaded}
\begin{Highlighting}[]
\FunctionTok{library}\NormalTok{(tidyverse)}
\end{Highlighting}
\end{Shaded}

\begin{verbatim}
## -- Attaching core tidyverse packages ------------------------ tidyverse 2.0.0 --
## v forcats   1.0.0     v purrr     1.0.2
## v ggplot2   3.5.1     v readr     2.1.5
## v lubridate 1.9.3     v tibble    3.2.1
\end{verbatim}

\begin{verbatim}
## -- Conflicts ------------------------------------------ tidyverse_conflicts() --
## x dplyr::filter() masks stats::filter()
## x dplyr::lag()    masks stats::lag()
## i Use the conflicted package (<http://conflicted.r-lib.org/>) to force all conflicts to become errors
\end{verbatim}

\begin{Shaded}
\begin{Highlighting}[]
\FunctionTok{library}\NormalTok{(ggplot2)}
\end{Highlighting}
\end{Shaded}

Load all the data

\begin{Shaded}
\begin{Highlighting}[]
\NormalTok{dataLink1 }\OtherTok{\textless{}{-}} \StringTok{"https://raw.githubusercontent.com/ErickH1/DATA607Project2/refs/heads/main/country\_results\_df.csv"}
\NormalTok{dataLink2 }\OtherTok{\textless{}{-}} \StringTok{"https://raw.githubusercontent.com/ErickH1/DATA607Project2/refs/heads/main/world\_population.csv"}
\NormalTok{dataLink3 }\OtherTok{\textless{}{-}} \StringTok{"https://raw.githubusercontent.com/ErickH1/DATA607Project2/refs/heads/main/MTA\_Daily\_Ridership.csv"}
\end{Highlighting}
\end{Shaded}

\subsection{Dataset 1 ---:}\label{dataset-1}

Description

\begin{Shaded}
\begin{Highlighting}[]
\NormalTok{country\_data }\OtherTok{\textless{}{-}} \FunctionTok{read.csv}\NormalTok{(dataLink1)}
\FunctionTok{head}\NormalTok{(country\_data)}
\end{Highlighting}
\end{Shaded}

\begin{verbatim}
##   year                    country team_size_all team_size_male team_size_female
## 1 2024   United States of America             6              5                1
## 2 2024 People's Republic of China             6              6                0
## 3 2024          Republic of Korea             6              6                0
## 4 2024                      India             6              6                0
## 5 2024                    Belarus             6              6                0
## 6 2024                  Singapore             6              6                0
##   p1 p2 p3 p4 p5 p6 p7 awards_gold awards_silver awards_bronze
## 1 42 41 19 40 35 15 NA           5             1             0
## 2 42 42 31 40 22 13 NA           5             1             0
## 3 42 37 18 42  7 22 NA           2             4             0
## 4 42 34 11 42 28 10 NA           4             1             0
## 5 42 30 10 42 36  5 NA           4             0             2
## 6 42 37  7 42 29  5 NA           1             5             0
##   awards_honorable_mentions                   leader    deputy_leader
## 1                         0              John Berman Carl Schildkraut
## 2                         0               Liang Xiao        Yijun Yao
## 3                         0             Suyoung Choi      Hwajong Yoo
## 4                         1 Krishnan Sivasubramanian      Rijul Saini
## 5                         0           David Zmiaikou  Dzmitry Bazyleu
## 6                         0           Yong Sheng Soh    Teck Kian Teo
\end{verbatim}

\subsection{Dataset 1 Tidy Data:}\label{dataset-1-tidy-data}

\begin{Shaded}
\begin{Highlighting}[]
\NormalTok{long\_data }\OtherTok{\textless{}{-}}\NormalTok{ country\_data }\SpecialCharTok{\%\textgreater{}\%}
  \FunctionTok{select}\NormalTok{(}\SpecialCharTok{{-}}\NormalTok{p1}\SpecialCharTok{:}\NormalTok{p7) }\SpecialCharTok{\%\textgreater{}\%}
  \FunctionTok{pivot\_longer}\NormalTok{(}\AttributeTok{cols =}\NormalTok{ p1}\SpecialCharTok{:}\NormalTok{p7, }\AttributeTok{names\_to =} \StringTok{"position"}\NormalTok{, }\AttributeTok{values\_to =} \StringTok{"score"}\NormalTok{)}
\end{Highlighting}
\end{Shaded}

\begin{verbatim}
## Warning in x:y: numerical expression has 17 elements: only the first used
\end{verbatim}

\begin{Shaded}
\begin{Highlighting}[]
\FunctionTok{head}\NormalTok{(long\_data)}
\end{Highlighting}
\end{Shaded}

\begin{verbatim}
## # A tibble: 6 x 7
##    year country     team_size_all team_size_male team_size_female position score
##   <int> <chr>               <int>          <int>            <int> <chr>    <int>
## 1  2024 United Sta~             6              5                1 p1          42
## 2  2024 United Sta~             6              5                1 p2          41
## 3  2024 United Sta~             6              5                1 p3          19
## 4  2024 United Sta~             6              5                1 p4          40
## 5  2024 United Sta~             6              5                1 p5          35
## 6  2024 United Sta~             6              5                1 p6          15
\end{verbatim}

\begin{Shaded}
\begin{Highlighting}[]
\NormalTok{total\_points }\OtherTok{\textless{}{-}}\NormalTok{ long\_data }\SpecialCharTok{\%\textgreater{}\%}
  \FunctionTok{group\_by}\NormalTok{(country, year) }\SpecialCharTok{\%\textgreater{}\%}
  \FunctionTok{summarise}\NormalTok{(}\AttributeTok{total\_points =} \FunctionTok{sum}\NormalTok{(score, }\AttributeTok{na.rm =} \ConstantTok{TRUE}\NormalTok{))}
\end{Highlighting}
\end{Shaded}

\begin{verbatim}
## `summarise()` has grouped output by 'country'. You can override using the
## `.groups` argument.
\end{verbatim}

\begin{Shaded}
\begin{Highlighting}[]
\NormalTok{tidy\_data }\OtherTok{\textless{}{-}}\NormalTok{ total\_points }\SpecialCharTok{\%\textgreater{}\%}
  \FunctionTok{select}\NormalTok{(country, year, total\_points)}

\NormalTok{tidy\_data\_filtered }\OtherTok{\textless{}{-}}\NormalTok{ tidy\_data }\SpecialCharTok{\%\textgreater{}\%}
  \FunctionTok{filter}\NormalTok{(year }\SpecialCharTok{\textgreater{}=} \DecValTok{2014} \SpecialCharTok{\&}\NormalTok{ year }\SpecialCharTok{\textless{}=} \DecValTok{2024}\NormalTok{)}

\NormalTok{tidy\_data\_filtered}\SpecialCharTok{$}\NormalTok{year }\OtherTok{\textless{}{-}} \FunctionTok{as.factor}\NormalTok{(tidy\_data\_filtered}\SpecialCharTok{$}\NormalTok{year)}

\FunctionTok{head}\NormalTok{(tidy\_data\_filtered,}\DecValTok{20}\NormalTok{)}
\end{Highlighting}
\end{Shaded}

\begin{verbatim}
## # A tibble: 20 x 3
## # Groups:   country [2]
##    country year  total_points
##    <chr>   <fct>        <int>
##  1 Albania 2014            46
##  2 Albania 2015            37
##  3 Albania 2016            58
##  4 Albania 2017            67
##  5 Albania 2018            37
##  6 Albania 2019            37
##  7 Albania 2020            40
##  8 Albania 2021            11
##  9 Albania 2022            62
## 10 Albania 2023            48
## 11 Albania 2024            51
## 12 Algeria 2015            60
## 13 Algeria 2016            41
## 14 Algeria 2017            70
## 15 Algeria 2018            18
## 16 Algeria 2019            46
## 17 Algeria 2020             5
## 18 Algeria 2021            16
## 19 Algeria 2022            75
## 20 Algeria 2023           100
\end{verbatim}

\begin{Shaded}
\begin{Highlighting}[]
\NormalTok{country\_totals }\OtherTok{\textless{}{-}}\NormalTok{ tidy\_data\_filtered }\SpecialCharTok{\%\textgreater{}\%}
  \FunctionTok{group\_by}\NormalTok{(country) }\SpecialCharTok{\%\textgreater{}\%}
  \FunctionTok{summarise}\NormalTok{(}\AttributeTok{total\_country\_points =} \FunctionTok{sum}\NormalTok{(total\_points)) }\SpecialCharTok{\%\textgreater{}\%}
  \FunctionTok{arrange}\NormalTok{(}\FunctionTok{desc}\NormalTok{(total\_country\_points))}

\NormalTok{top\_10\_countries }\OtherTok{\textless{}{-}}\NormalTok{ country\_totals }\SpecialCharTok{\%\textgreater{}\%}
  \FunctionTok{slice}\NormalTok{(}\DecValTok{1}\SpecialCharTok{:}\DecValTok{10}\NormalTok{) }\SpecialCharTok{\%\textgreater{}\%}
  \FunctionTok{pull}\NormalTok{(country)}

\NormalTok{tidy\_data\_filtered\_top\_10 }\OtherTok{\textless{}{-}}\NormalTok{ tidy\_data\_filtered }\SpecialCharTok{\%\textgreater{}\%}
  \FunctionTok{filter}\NormalTok{(country }\SpecialCharTok{\%in\%}\NormalTok{ top\_10\_countries)}

\FunctionTok{head}\NormalTok{(tidy\_data\_filtered\_top\_10,}\DecValTok{40}\NormalTok{)}
\end{Highlighting}
\end{Shaded}

\begin{verbatim}
## # A tibble: 40 x 3
## # Groups:   country [4]
##    country year  total_points
##    <chr>   <fct>        <int>
##  1 Canada  2014           159
##  2 Canada  2015           140
##  3 Canada  2016           148
##  4 Canada  2017           110
##  5 Canada  2018           156
##  6 Canada  2019           144
##  7 Canada  2020           161
##  8 Canada  2021           151
##  9 Canada  2022           178
## 10 Canada  2023           183
## # i 30 more rows
\end{verbatim}

\subsection{Dataset 1 Data Analysis:}\label{dataset-1-data-analysis}

\begin{Shaded}
\begin{Highlighting}[]
\CommentTok{\# Create the plot}
\FunctionTok{ggplot}\NormalTok{(tidy\_data\_filtered\_top\_10, }\FunctionTok{aes}\NormalTok{(}\AttributeTok{x =}\NormalTok{ year, }\AttributeTok{y =}\NormalTok{ total\_points, }\AttributeTok{color =}\NormalTok{ country, }\AttributeTok{group =}\NormalTok{ country)) }\SpecialCharTok{+}
  \FunctionTok{geom\_line}\NormalTok{() }\SpecialCharTok{+}
  \FunctionTok{labs}\NormalTok{(}\AttributeTok{title =} \StringTok{"Total Points by Country (Last 10 Years)"}\NormalTok{,}
       \AttributeTok{x =} \StringTok{"Year"}\NormalTok{,}
       \AttributeTok{y =} \StringTok{"Total Points"}\NormalTok{,}
       \AttributeTok{color =} \StringTok{"Country"}\NormalTok{) }\SpecialCharTok{+}
  \FunctionTok{theme\_minimal}\NormalTok{()}
\end{Highlighting}
\end{Shaded}

\includegraphics{DATA607Project2_files/figure-latex/unnamed-chunk-7-1.pdf}
\#\# Dataset 1 Conclusion: In summary the data was tidied to only
include observations for total points over the years for the top 20
countries in the Math Olympiad. The data was then graphed which each
line observing the countries and the total points. We can see that the
countries have the same dips and peaks indicating potential difficult
and easier years. It also showcases which country performed the best
(China) and worst within the top 20 (Thailand)

\subsection{Dataset 2 ---:}\label{dataset-2}

Description

\begin{Shaded}
\begin{Highlighting}[]
\NormalTok{world\_data }\OtherTok{\textless{}{-}} \FunctionTok{read.csv}\NormalTok{(dataLink2)}
\FunctionTok{head}\NormalTok{(world\_data)}
\end{Highlighting}
\end{Shaded}

\begin{verbatim}
##   Rank CCA3 Country.Territory          Capital Continent X2022.Population
## 1   36  AFG       Afghanistan            Kabul      Asia         41128771
## 2  138  ALB           Albania           Tirana    Europe          2842321
## 3   34  DZA           Algeria          Algiers    Africa         44903225
## 4  213  ASM    American Samoa        Pago Pago   Oceania            44273
## 5  203  AND           Andorra Andorra la Vella    Europe            79824
## 6   42  AGO            Angola           Luanda    Africa         35588987
##   X2020.Population X2015.Population X2010.Population X2000.Population
## 1         38972230         33753499         28189672         19542982
## 2          2866849          2882481          2913399          3182021
## 3         43451666         39543154         35856344         30774621
## 4            46189            51368            54849            58230
## 5            77700            71746            71519            66097
## 6         33428485         28127721         23364185         16394062
##   X1990.Population X1980.Population X1970.Population Area..km..
## 1         10694796         12486631         10752971     652230
## 2          3295066          2941651          2324731      28748
## 3         25518074         18739378         13795915    2381741
## 4            47818            32886            27075        199
## 5            53569            35611            19860        468
## 6         11828638          8330047          6029700    1246700
##   Density..per.km.. Growth.Rate World.Population.Percentage
## 1           63.0587      1.0257                        0.52
## 2           98.8702      0.9957                        0.04
## 3           18.8531      1.0164                        0.56
## 4          222.4774      0.9831                        0.00
## 5          170.5641      1.0100                        0.00
## 6           28.5466      1.0315                        0.45
\end{verbatim}

\subsection{Dataset 2 Tidy Data:}\label{dataset-2-tidy-data}

\begin{Shaded}
\begin{Highlighting}[]
\NormalTok{world\_data }\OtherTok{\textless{}{-}}\NormalTok{ world\_data }\SpecialCharTok{\%\textgreater{}\%}
  \FunctionTok{rename}\NormalTok{(}\StringTok{"Country/Territory"} \OtherTok{=}\NormalTok{ Country.Territory, }\StringTok{"2022"} \OtherTok{=}\NormalTok{ X2022.Population, }\StringTok{"2020"} \OtherTok{=}\NormalTok{ X2020.Population, }\StringTok{"2015"} \OtherTok{=}\NormalTok{ X2015.Population, }\StringTok{"2010"} \OtherTok{=}\NormalTok{ X2010.Population, }\StringTok{"2000"} \OtherTok{=}\NormalTok{ X2000.Population, }\StringTok{"1990"} \OtherTok{=}\NormalTok{ X1990.Population, }\StringTok{"1980"} \OtherTok{=}\NormalTok{ X1980.Population, }\StringTok{"1970"} \OtherTok{=}\NormalTok{ X1970.Population, }\StringTok{"Area (km)"} \OtherTok{=}\NormalTok{ Area..km.., }\StringTok{"Density per km"} \OtherTok{=}\NormalTok{ Density..per.km.., }\StringTok{"Growth Rate"} \OtherTok{=}\NormalTok{ Growth.Rate, }\StringTok{"World Population Percentage"} \OtherTok{=}\NormalTok{ World.Population.Percentage)}

\NormalTok{world\_data\_long }\OtherTok{\textless{}{-}}\NormalTok{ world\_data }\SpecialCharTok{\%\textgreater{}\%}
  \FunctionTok{pivot\_longer}\NormalTok{(}\StringTok{\textasciigrave{}}\AttributeTok{2022}\StringTok{\textasciigrave{}}\SpecialCharTok{:}\StringTok{\textasciigrave{}}\AttributeTok{1970}\StringTok{\textasciigrave{}}\NormalTok{, }\AttributeTok{names\_to =} \StringTok{"Year"}\NormalTok{, }\AttributeTok{values\_to =} \StringTok{"Population"}\NormalTok{)}

\FunctionTok{head}\NormalTok{(world\_data\_long)}
\end{Highlighting}
\end{Shaded}

\begin{verbatim}
## # A tibble: 6 x 11
##    Rank CCA3  `Country/Territory` Capital Continent `Area (km)` `Density per km`
##   <int> <chr> <chr>               <chr>   <chr>           <int>            <dbl>
## 1    36 AFG   Afghanistan         Kabul   Asia           652230             63.1
## 2    36 AFG   Afghanistan         Kabul   Asia           652230             63.1
## 3    36 AFG   Afghanistan         Kabul   Asia           652230             63.1
## 4    36 AFG   Afghanistan         Kabul   Asia           652230             63.1
## 5    36 AFG   Afghanistan         Kabul   Asia           652230             63.1
## 6    36 AFG   Afghanistan         Kabul   Asia           652230             63.1
## # i 4 more variables: `Growth Rate` <dbl>, `World Population Percentage` <dbl>,
## #   Year <chr>, Population <int>
\end{verbatim}

\begin{Shaded}
\begin{Highlighting}[]
\NormalTok{world\_data\_tidy }\OtherTok{\textless{}{-}}\NormalTok{ world\_data\_long }\SpecialCharTok{\%\textgreater{}\%}
  \FunctionTok{group\_by}\NormalTok{(Year, Continent) }\SpecialCharTok{\%\textgreater{}\%}
  \FunctionTok{summarise}\NormalTok{(}\AttributeTok{Total\_Population =} \FunctionTok{sum}\NormalTok{(Population))}
\end{Highlighting}
\end{Shaded}

\begin{verbatim}
## `summarise()` has grouped output by 'Year'. You can override using the
## `.groups` argument.
\end{verbatim}

\begin{Shaded}
\begin{Highlighting}[]
\NormalTok{world\_data\_tidy}\SpecialCharTok{$}\NormalTok{Year }\OtherTok{\textless{}{-}} \FunctionTok{as.numeric}\NormalTok{(}\FunctionTok{as.character}\NormalTok{(world\_data\_tidy}\SpecialCharTok{$}\NormalTok{Year))}

\FunctionTok{glimpse}\NormalTok{(world\_data\_tidy)}
\end{Highlighting}
\end{Shaded}

\begin{verbatim}
## Rows: 48
## Columns: 3
## Groups: Year [8]
## $ Year             <dbl> 1970, 1970, 1970, 1970, 1970, 1970, 1980, 1980, 1980,~
## $ Continent        <chr> "Africa", "Asia", "Europe", "North America", "Oceania~
## $ Total_Population <dbl> 365444348, 2144906290, 655923991, 315434606, 19480270~
\end{verbatim}

\subsection{Dataset 2 Data Analysis:}\label{dataset-2-data-analysis}

\begin{Shaded}
\begin{Highlighting}[]
\FunctionTok{ggplot}\NormalTok{(}\AttributeTok{data =}\NormalTok{ world\_data\_tidy, }\FunctionTok{aes}\NormalTok{(}\AttributeTok{x =}\NormalTok{ Year, }\AttributeTok{y =}\NormalTok{ Total\_Population, }\AttributeTok{color =}\NormalTok{ Continent)) }\SpecialCharTok{+}
  \FunctionTok{geom\_line}\NormalTok{(}\AttributeTok{linewidth =} \DecValTok{1}\NormalTok{) }\SpecialCharTok{+}
  \FunctionTok{ggtitle}\NormalTok{(}\StringTok{"World Population Over Time by Continent"}\NormalTok{) }\SpecialCharTok{+}
  \FunctionTok{xlab}\NormalTok{(}\StringTok{"Year"}\NormalTok{) }\SpecialCharTok{+}
  \FunctionTok{ylab}\NormalTok{(}\StringTok{"Total Population"}\NormalTok{) }\SpecialCharTok{+}
  \FunctionTok{theme\_minimal}\NormalTok{()}
\end{Highlighting}
\end{Shaded}

\includegraphics{DATA607Project2_files/figure-latex/unnamed-chunk-11-1.pdf}
\#\# Dataset 2 Conclusion: In conclusion the data was tidied to showcase
only the total population for each continent over the years. The data
was then graphed on a line graph. It showcases the countries with the
high total population (Asia) and the lowest (Oceania). It also showcases
the trends and slope throguhtout the years. \#\# Dataset 3 ---:
Description

\begin{Shaded}
\begin{Highlighting}[]
\NormalTok{mta\_data }\OtherTok{\textless{}{-}} \FunctionTok{read\_csv}\NormalTok{(dataLink3)}
\end{Highlighting}
\end{Shaded}

\begin{verbatim}
## Rows: 1671 Columns: 15
## -- Column specification --------------------------------------------------------
## Delimiter: ","
## chr  (1): Date
## dbl (14): Subways: Total Estimated Ridership, Subways: % of Comparable Pre-P...
## 
## i Use `spec()` to retrieve the full column specification for this data.
## i Specify the column types or set `show_col_types = FALSE` to quiet this message.
\end{verbatim}

\begin{Shaded}
\begin{Highlighting}[]
\FunctionTok{head}\NormalTok{(mta\_data)}
\end{Highlighting}
\end{Shaded}

\begin{verbatim}
## # A tibble: 6 x 15
##   Date   Subways: Total Estimate~1 Subways: % of Compar~2 Buses: Total Estimat~3
##   <chr>                      <dbl>                  <dbl>                  <dbl>
## 1 3/1/20                   2212965                     97                 984908
## 2 3/2/20                   5329915                     96                2209066
## 3 3/3/20                   5481103                     98                2228608
## 4 3/4/20                   5498809                     99                2177165
## 5 3/5/20                   5496453                     99                2244515
## 6 3/6/20                   5189447                     93                2066743
## # i abbreviated names: 1: `Subways: Total Estimated Ridership`,
## #   2: `Subways: % of Comparable Pre-Pandemic Day`,
## #   3: `Buses: Total Estimated Ridership`
## # i 11 more variables: `Buses: % of Comparable Pre-Pandemic Day` <dbl>,
## #   `LIRR: Total Estimated Ridership` <dbl>,
## #   `LIRR: % of Comparable Pre-Pandemic Day` <dbl>,
## #   `Metro-North: Total Estimated Ridership` <dbl>, ...
\end{verbatim}

\subsection{Dataset 3 Tidy Data:}\label{dataset-3-tidy-data}

\begin{Shaded}
\begin{Highlighting}[]
\CommentTok{\# Change from wide to long data }
\NormalTok{mta\_data\_long }\OtherTok{\textless{}{-}}\NormalTok{ mta\_data }\SpecialCharTok{\%\textgreater{}\%}
  \FunctionTok{pivot\_longer}\NormalTok{(}\SpecialCharTok{{-}}\NormalTok{Date, }\AttributeTok{names\_to =}\FunctionTok{c}\NormalTok{(}\StringTok{"transportation"}\NormalTok{,}\StringTok{"percent\_prepandemic"}\NormalTok{),}
               \AttributeTok{names\_sep =} \StringTok{":"}\NormalTok{, ) }\SpecialCharTok{\%\textgreater{}\%}
  \FunctionTok{pivot\_wider}\NormalTok{(}\AttributeTok{names\_from =}\NormalTok{ percent\_prepandemic, }\AttributeTok{values\_from =}\NormalTok{ value)}

\NormalTok{mta\_data\_long }\OtherTok{\textless{}{-}}\NormalTok{ mta\_data\_long }\SpecialCharTok{\%\textgreater{}\%}
  \FunctionTok{select}\NormalTok{(}\SpecialCharTok{{-}}\FunctionTok{c}\NormalTok{(}\StringTok{\textasciigrave{}}\AttributeTok{ Total Scheduled Trips}\StringTok{\textasciigrave{}}\NormalTok{, }\StringTok{\textasciigrave{}}\AttributeTok{ Total Traffic}\StringTok{\textasciigrave{}}\NormalTok{))}

\NormalTok{mta\_data\_long }\OtherTok{\textless{}{-}}\NormalTok{ mta\_data\_long }\SpecialCharTok{\%\textgreater{}\%}
  \FunctionTok{rename}\NormalTok{(}\AttributeTok{total\_ridership =} \StringTok{\textasciigrave{}}\AttributeTok{ Total Estimated Ridership}\StringTok{\textasciigrave{}}\NormalTok{)}

\FunctionTok{head}\NormalTok{(mta\_data\_long)}
\end{Highlighting}
\end{Shaded}

\begin{verbatim}
## # A tibble: 6 x 4
##   Date   transportation      total_ridership ` % of Comparable Pre-Pandemic Day`
##   <chr>  <chr>                         <dbl>                               <dbl>
## 1 3/1/20 Subways                     2212965                                  97
## 2 3/1/20 Buses                        984908                                  99
## 3 3/1/20 LIRR                          86790                                 100
## 4 3/1/20 Metro-North                   55825                                  59
## 5 3/1/20 Access-A-Ride                    NA                                 113
## 6 3/1/20 Bridges and Tunnels              NA                                  98
\end{verbatim}

\begin{Shaded}
\begin{Highlighting}[]
\FunctionTok{glimpse}\NormalTok{(mta\_data\_long)}
\end{Highlighting}
\end{Shaded}

\begin{verbatim}
## Rows: 11,697
## Columns: 4
## $ Date                                <chr> "3/1/20", "3/1/20", "3/1/20", "3/1~
## $ transportation                      <chr> "Subways", "Buses", "LIRR", "Metro~
## $ total_ridership                     <dbl> 2212965, 984908, 86790, 55825, NA,~
## $ ` % of Comparable Pre-Pandemic Day` <dbl> 97, 99, 100, 59, 113, 98, 52, 96, ~
\end{verbatim}

\begin{Shaded}
\begin{Highlighting}[]
\CommentTok{\# Convert the date column to a proper date format}
\NormalTok{mta\_data\_long}\SpecialCharTok{$}\NormalTok{Date }\OtherTok{\textless{}{-}} \FunctionTok{as.Date}\NormalTok{(mta\_data\_long}\SpecialCharTok{$}\NormalTok{Date, }\AttributeTok{format =} \StringTok{"\%m/\%d/\%y"}\NormalTok{)}

\CommentTok{\# Extract the year from the date column}
\NormalTok{mta\_data\_long}\SpecialCharTok{$}\NormalTok{Year }\OtherTok{\textless{}{-}} \FunctionTok{as.numeric}\NormalTok{(}\FunctionTok{format}\NormalTok{(mta\_data\_long}\SpecialCharTok{$}\NormalTok{Date, }\StringTok{"\%Y"}\NormalTok{))}

\CommentTok{\# View the first few rows of the updated DataFrame}
\FunctionTok{head}\NormalTok{(mta\_data\_long)}
\end{Highlighting}
\end{Shaded}

\begin{verbatim}
## # A tibble: 6 x 5
##   Date       transportation      total_ridership  % of Comparable Pre-Pa~1  Year
##   <date>     <chr>                         <dbl>                     <dbl> <dbl>
## 1 2020-03-01 Subways                     2212965                        97  2020
## 2 2020-03-01 Buses                        984908                        99  2020
## 3 2020-03-01 LIRR                          86790                       100  2020
## 4 2020-03-01 Metro-North                   55825                        59  2020
## 5 2020-03-01 Access-A-Ride                    NA                       113  2020
## 6 2020-03-01 Bridges and Tunnels              NA                        98  2020
## # i abbreviated name: 1: ` % of Comparable Pre-Pandemic Day`
\end{verbatim}

\begin{Shaded}
\begin{Highlighting}[]
\NormalTok{mta\_data\_tidy }\OtherTok{\textless{}{-}}\NormalTok{ mta\_data\_long }\SpecialCharTok{\%\textgreater{}\%}
  \FunctionTok{group\_by}\NormalTok{(Year, transportation) }\SpecialCharTok{\%\textgreater{}\%}
  \FunctionTok{summarize}\NormalTok{(}\AttributeTok{mean\_total\_ride =} \FunctionTok{mean}\NormalTok{(total\_ridership))}
\end{Highlighting}
\end{Shaded}

\begin{verbatim}
## `summarise()` has grouped output by 'Year'. You can override using the
## `.groups` argument.
\end{verbatim}

\begin{Shaded}
\begin{Highlighting}[]
\FunctionTok{glimpse}\NormalTok{(mta\_data\_tidy)}
\end{Highlighting}
\end{Shaded}

\begin{verbatim}
## Rows: 35
## Columns: 3
## Groups: Year [5]
## $ Year            <dbl> 2020, 2020, 2020, 2020, 2020, 2020, 2020, 2021, 2021, ~
## $ transportation  <chr> "Access-A-Ride", "Bridges and Tunnels", "Buses", "LIRR~
## $ mean_total_ride <dbl> NA, NA, 481659.147, 58224.588, 37544.441, 2359.206, 12~
\end{verbatim}

\subsection{Dataset 3 Data Analysis:}\label{dataset-3-data-analysis}

\begin{Shaded}
\begin{Highlighting}[]
\FunctionTok{ggplot}\NormalTok{(}\AttributeTok{data =}\NormalTok{ mta\_data\_tidy, }\FunctionTok{aes}\NormalTok{(}\AttributeTok{x =} \FunctionTok{factor}\NormalTok{(Year), }\AttributeTok{y =}\NormalTok{ mean\_total\_ride, }\AttributeTok{fill =}\NormalTok{ transportation)) }\SpecialCharTok{+}
  \FunctionTok{geom\_bar}\NormalTok{(}\AttributeTok{stat =} \StringTok{"identity"}\NormalTok{, }\AttributeTok{position =} \StringTok{"dodge"}\NormalTok{) }\SpecialCharTok{+}
  \FunctionTok{ggtitle}\NormalTok{(}\StringTok{"Average Ridership over the Years by transport"}\NormalTok{) }\SpecialCharTok{+}
  \FunctionTok{xlab}\NormalTok{(}\StringTok{"Year"}\NormalTok{) }\SpecialCharTok{+}
  \FunctionTok{ylab}\NormalTok{(}\StringTok{"Average Total Riders"}\NormalTok{) }\SpecialCharTok{+}
  \FunctionTok{theme\_minimal}\NormalTok{() }\SpecialCharTok{+}
  \FunctionTok{scale\_x\_discrete}\NormalTok{(}\AttributeTok{labels =} \FunctionTok{c}\NormalTok{(}\StringTok{"2020"}\NormalTok{, }\StringTok{"2021"}\NormalTok{, }\StringTok{"2022"}\NormalTok{, }\StringTok{"2023"}\NormalTok{, }\StringTok{"2024"}\NormalTok{))}
\end{Highlighting}
\end{Shaded}

\begin{verbatim}
## Warning: Removed 10 rows containing missing values or values outside the scale range
## (`geom_bar()`).
\end{verbatim}

\includegraphics{DATA607Project2_files/figure-latex/unnamed-chunk-15-1.pdf}
\#\# Dataset 3 Conclusion: In conclusion the data was tidied by only
including the Average total riders over the years based on
transportation method. The mean ridership was then calculated based on
the transportation and then graphed. The graph showcases the yearly
trend and which transport had the most (Subway) and which had the least
(hard to tell because of large difference but it is either acces a ride
or bridges and tunnels)

\end{document}
